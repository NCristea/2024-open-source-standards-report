% Options for packages loaded elsewhere
\PassOptionsToPackage{unicode}{hyperref}
\PassOptionsToPackage{hyphens}{url}
\PassOptionsToPackage{dvipsnames,svgnames,x11names}{xcolor}
%
\documentclass[
  letterpaper,
  DIV=11,
  numbers=noendperiod]{scrartcl}

\usepackage{amsmath,amssymb}
\usepackage{iftex}
\ifPDFTeX
  \usepackage[T1]{fontenc}
  \usepackage[utf8]{inputenc}
  \usepackage{textcomp} % provide euro and other symbols
\else % if luatex or xetex
  \usepackage{unicode-math}
  \defaultfontfeatures{Scale=MatchLowercase}
  \defaultfontfeatures[\rmfamily]{Ligatures=TeX,Scale=1}
\fi
\usepackage{lmodern}
\ifPDFTeX\else  
    % xetex/luatex font selection
\fi
% Use upquote if available, for straight quotes in verbatim environments
\IfFileExists{upquote.sty}{\usepackage{upquote}}{}
\IfFileExists{microtype.sty}{% use microtype if available
  \usepackage[]{microtype}
  \UseMicrotypeSet[protrusion]{basicmath} % disable protrusion for tt fonts
}{}
\makeatletter
\@ifundefined{KOMAClassName}{% if non-KOMA class
  \IfFileExists{parskip.sty}{%
    \usepackage{parskip}
  }{% else
    \setlength{\parindent}{0pt}
    \setlength{\parskip}{6pt plus 2pt minus 1pt}}
}{% if KOMA class
  \KOMAoptions{parskip=half}}
\makeatother
\usepackage{xcolor}
\setlength{\emergencystretch}{3em} % prevent overfull lines
\setcounter{secnumdepth}{5}
% Make \paragraph and \subparagraph free-standing
\ifx\paragraph\undefined\else
  \let\oldparagraph\paragraph
  \renewcommand{\paragraph}[1]{\oldparagraph{#1}\mbox{}}
\fi
\ifx\subparagraph\undefined\else
  \let\oldsubparagraph\subparagraph
  \renewcommand{\subparagraph}[1]{\oldsubparagraph{#1}\mbox{}}
\fi


\providecommand{\tightlist}{%
  \setlength{\itemsep}{0pt}\setlength{\parskip}{0pt}}\usepackage{longtable,booktabs,array}
\usepackage{calc} % for calculating minipage widths
% Correct order of tables after \paragraph or \subparagraph
\usepackage{etoolbox}
\makeatletter
\patchcmd\longtable{\par}{\if@noskipsec\mbox{}\fi\par}{}{}
\makeatother
% Allow footnotes in longtable head/foot
\IfFileExists{footnotehyper.sty}{\usepackage{footnotehyper}}{\usepackage{footnote}}
\makesavenoteenv{longtable}
\usepackage{graphicx}
\makeatletter
\def\maxwidth{\ifdim\Gin@nat@width>\linewidth\linewidth\else\Gin@nat@width\fi}
\def\maxheight{\ifdim\Gin@nat@height>\textheight\textheight\else\Gin@nat@height\fi}
\makeatother
% Scale images if necessary, so that they will not overflow the page
% margins by default, and it is still possible to overwrite the defaults
% using explicit options in \includegraphics[width, height, ...]{}
\setkeys{Gin}{width=\maxwidth,height=\maxheight,keepaspectratio}
% Set default figure placement to htbp
\makeatletter
\def\fps@figure{htbp}
\makeatother
% definitions for citeproc citations
\NewDocumentCommand\citeproctext{}{}
\NewDocumentCommand\citeproc{mm}{%
  \begingroup\def\citeproctext{#2}\cite{#1}\endgroup}
\makeatletter
 % allow citations to break across lines
 \let\@cite@ofmt\@firstofone
 % avoid brackets around text for \cite:
 \def\@biblabel#1{}
 \def\@cite#1#2{{#1\if@tempswa , #2\fi}}
\makeatother
\newlength{\cslhangindent}
\setlength{\cslhangindent}{1.5em}
\newlength{\csllabelwidth}
\setlength{\csllabelwidth}{3em}
\newenvironment{CSLReferences}[2] % #1 hanging-indent, #2 entry-spacing
 {\begin{list}{}{%
  \setlength{\itemindent}{0pt}
  \setlength{\leftmargin}{0pt}
  \setlength{\parsep}{0pt}
  % turn on hanging indent if param 1 is 1
  \ifodd #1
   \setlength{\leftmargin}{\cslhangindent}
   \setlength{\itemindent}{-1\cslhangindent}
  \fi
  % set entry spacing
  \setlength{\itemsep}{#2\baselineskip}}}
 {\end{list}}
\usepackage{calc}
\newcommand{\CSLBlock}[1]{\hfill\break\parbox[t]{\linewidth}{\strut\ignorespaces#1\strut}}
\newcommand{\CSLLeftMargin}[1]{\parbox[t]{\csllabelwidth}{\strut#1\strut}}
\newcommand{\CSLRightInline}[1]{\parbox[t]{\linewidth - \csllabelwidth}{\strut#1\strut}}
\newcommand{\CSLIndent}[1]{\hspace{\cslhangindent}#1}

\KOMAoption{captions}{tableheading}
\makeatletter
\@ifpackageloaded{caption}{}{\usepackage{caption}}
\AtBeginDocument{%
\ifdefined\contentsname
  \renewcommand*\contentsname{Table of contents}
\else
  \newcommand\contentsname{Table of contents}
\fi
\ifdefined\listfigurename
  \renewcommand*\listfigurename{List of Figures}
\else
  \newcommand\listfigurename{List of Figures}
\fi
\ifdefined\listtablename
  \renewcommand*\listtablename{List of Tables}
\else
  \newcommand\listtablename{List of Tables}
\fi
\ifdefined\figurename
  \renewcommand*\figurename{Figure}
\else
  \newcommand\figurename{Figure}
\fi
\ifdefined\tablename
  \renewcommand*\tablename{Table}
\else
  \newcommand\tablename{Table}
\fi
}
\@ifpackageloaded{float}{}{\usepackage{float}}
\floatstyle{ruled}
\@ifundefined{c@chapter}{\newfloat{codelisting}{h}{lop}}{\newfloat{codelisting}{h}{lop}[chapter]}
\floatname{codelisting}{Listing}
\newcommand*\listoflistings{\listof{codelisting}{List of Listings}}
\makeatother
\makeatletter
\makeatother
\makeatletter
\@ifpackageloaded{caption}{}{\usepackage{caption}}
\@ifpackageloaded{subcaption}{}{\usepackage{subcaption}}
\makeatother
\ifLuaTeX
  \usepackage{selnolig}  % disable illegal ligatures
\fi
\usepackage{bookmark}

\IfFileExists{xurl.sty}{\usepackage{xurl}}{} % add URL line breaks if available
\urlstyle{same} % disable monospaced font for URLs
\hypersetup{
  pdftitle={Towards an open-source model for data and metadata standards},
  colorlinks=true,
  linkcolor={blue},
  filecolor={Maroon},
  citecolor={Blue},
  urlcolor={Blue},
  pdfcreator={LaTeX via pandoc}}

\title{Towards an open-source model for data and metadata standards}
\author{Ariel Rokem \and Vani Mandava}
\date{}

\begin{document}
\maketitle

\section{🚧 Under construction 🚧}\label{under-construction}

Please excuse our dust while we work on this report, which is currently
under heavy construction.

\section{Abstract}\label{abstract}

Recent progress in machine learning and artificial intelligence promises
to advance research and understanding across a wide range of fields and
activities. In tandem, increased awareness of the importance of open
data for reproducibility and scientific transparency is making inroads
in fields that have not traditionally produced large publicly available
datasets. Data sharing requirements from publishers and funders, as well
as from other stakeholders, have also created pressure to make datasets
with research and/or public interest value available through digital
repositories. However, to make the best use of existing data, and
facilitate the creation of useful future datasets, robust, interoperable
and usable standards need to evolve and adapt over time. The open-source
development model provides significant potential benefits to the process
of standard creation and adaptation. In particular, the development and
adaptation of standards can use long-standing socio-technical processes
that have been key to managing the development of software, and allow
incorporating broad community input into the formulation of these
standards. By adhering to open-source standards to formal descriptions
(e.g., by implementing schemata for standard specification, and/or by
implementing automated standard validation), processes such as automated
testing and continuous integration, which have been important in the
development of open-source software, can be adopted in defining data and
metadata standards as well. Similarly, open-source governance provides a
range of stakeholders a voice in the development of standards,
potentially enabling use cases and concerns that would not be taken into
account in a top-down model of standards development. On the other hand,
open-source models carry unique risks that need to be incorporated into
the process.

\section{Introduction}\label{introduction}

Data-intensive discovery has become an important mode of knowledge
production across many research fields and it is having a significant
and broad impact across all of society. This is becoming increasingly
salient as recent developments in machine learning and artificial
intelligence (AI) promise to increase the value of large,
multi-dimensional, heterogeneous data sources. Coupled with these new
machine learning techniques, these datasets can help us understand
everything from the cellular operations of the human body, through
business transactions on the internet, to the structure and history of
the universe. However, the development of new machine learning methods
and data-intensive discovery more generally depends on Findability,
Accessibility, Interoperability and Reusability (FAIR) of data
(Wilkinson et al. 2016). One of the main mechanisms through which the
FAIR principles are promoted is the development of \emph{standards} for
data and metadata. Standards can vary in the level of detail and scope,
and encompass such things as \emph{file formats} for the storage of
certain data types, \emph{schemas} for databases that organize data,
\emph{ontologies} to describe and organize metadata in a manner that
connects it to field-specific meaning, as well as mechanisms to describe
\emph{provenance} of analysis products.

Community-driven development of robust, adaptable and useful standards
draws significant inspiration from the development of open-source
software (OSS) and has many parallels and overlaps with OSS development.
OSS has a long history going back to the development of the Unix
operating system in the late 1960s. Over the time since its inception,
the large community of developers and users of OSS have developed a host
of socio-technical mechanisms that support the development and use of
OSS. For example, the Open Source Initiative (OSI), a non-profit
organization that was founded in the 1990s developed a set of guidelines
for licensing of OSS that is designed to protect the rights of
developers and users. On the more technical side, tools such as the Git
Source-code management system support open-source development workflows
that can be adopted in the development of standards. Governance
approaches have been honed to address the challenges of managing a range
of stakeholder interests and to mediate between large numbers of
weakly-connected individuals that contribute to OSS. When these social
and technical innovations are put together they enable a host of
positive defining features of OSS, such as transparency, collaboration,
and decentralization. These features allow OSS to have a remarkable
level of dynamism and productivity, while also retaining the ability of
a variety of stakeholders to guide the evolution of the software to take
their needs and interests into account.

The present report seeks to explore how OSS processes and tools have
affected the development of data and metadata standards. The report will
triangulate common features of a variety of use cases; it will identify
some of the challenges and pitfalls of this mode of standards
development; and it will make recommendations for future developments
and policies that can help this mode of standards development thrive and
reach its full potential.

\section{Opportunities and risks for open-source
standards}\label{opportunities-and-risks-for-open-source-standards}

Data and metadata standards that adopt tools and practices of OSS
(``open-source standards'' henceforth) stand to reap many of the
benefits that the OSS model has provided in the development of other
technologies. At the same time, these tools and practices are associated
with risks that need to be mitigated.

\subsection{Flexibility vs.~stability}\label{flexibility-vs.-stability}

One of the defining characteristics of OSS is its dynamism and its rapid
evolution. Because OSS can be used by anyone and, in most cases,
contributions can be made by anyone, innovations flow into OSS in a
bottom-up fashion from user/developers. Pathways to contribution by
members of the community are often well-defined: both from the technical
perspective (e.g., through a pull request on GitHub, or other similar
mechanisms), as well as from the social perspective (e.g., whether
contributors need to accept certain licensing conditions through a
contributor licensing agreement) and the socio-technical perspective
(e.g., how many people need to review a contribution, what are the
timelines for a contribution to be reviewed and accepted, what are the
release cycles of the software that make the contribution available to a
broader community of users, etc.). Similarly, open-source standards may
also find themselves addressing use cases and solutions that were not
originally envisioned through bottom-up contributions of members of a
research community to which the standard pertains. However, while this
dynamism provides an avenue for flexibility it also presents a source of
tension. This is because data and metadata standards apply to already
existing datasets, and changes may affect the compliance of these
existing datasets.

\subsection{Mismatches between standards developers and user
communities}\label{mismatches-between-standards-developers-and-user-communities}

There is an inherent gap in both interest and ability to engage with the
technical details undergirding standards and their development between
the core developers of the standard and their users. In extreme cases,
these interests may even be at odds, as developers implement
sophisticated mechanisms to automate the creation of the standard or
advocate for more technically advanced mechanisms for evolving the
standard, leaving potential users sidelined in the development of the
standard, and limiting their ability to provide feedback about the
practical implications of changes to the standards.

\subsection{Unclear pathways for standards
success}\label{unclear-pathways-for-standards-success}

Standards typically develop organically through sustained and persistent
efforts from dedicated groups of data practitioneers. These include
scientists and the broader ecosystem of data curators and users. However
there is no playbook on the structure and components of a data standard,
or the pathway that moves a data implementation to a data standard. As a
result, data standardization lacks formal avenues for research grants.

\subsection{Cross domain funding gaps}\label{cross-domain-funding-gaps}

Data standardization investment is justified if the standard is
generalizable beyond any specific science domain. However while the use
cases are domain sciences based, data standardization is seen as a data
infrastructure and not a science investment. Moreover due to how science
research funding works, scientists lack incentives to work across
domains, or work on infrastructure problems.

\subsection{Data instrumentation
issues}\label{data-instrumentation-issues}

Data for scientific observations are often generated by proprietary
instrumentation due to commercialization or other profit driven
incentives. There islack of regulatory oversight to adhere to available
standards or evolve Significant data transformation is required to get
data to a state that is amenable to standards, if available. If not
available, there is lack of incentive to set aside investment or
resources to invest in establishing data standards.

\subsection{Sustainability}\label{sustainability}

\subsection{The importance of automated
validation}\label{the-importance-of-automated-validation}

\section{Use cases}\label{use-cases}

To understand how OSS development practices affect the development of
data and metadata standards, it is informative to demonstrate this
cross-fertilization through a few use cases. As we will see in these
examples some fields, such as astronomy, high-energy physics and earth
sciences have a relatively long history of shared data resources from
organizations such as LSST and CERN, while other fields have only
relatively recently become aware of the value of data sharing and its
impact. These disparate histories inform how standards have evolved and
how OSS practices have pervaded their development.

\subsection{Astronomy}\label{astronomy}

One prominent example of a community-driven standard is the FITS
(Flexible Image Transport System) file format standard, which was
developed in the late 1970s and early 1980s (Wells and Greisen 1979),
and has been adopted worldwide for astronomy data preservation and
exchange. Essentially every software platform used in astronomy reads
and writes the FITS format. It was developed by observatories in the
1980s to store image data in the visible and x-ray spectrum. It has been
endorsed by IAU, as well as funding agencies. Though the format has
evolved over time, ``once FITS, always FITS''. That is, the format
cannot be evolved to introduce changes that break backwards
compatibility. Among the features that make FITS so durable is that it
was designed originally to have a very restricted metadata schema. That
is, FITS records were designed to be the lowest common denominator of
word lengths in computer systems at the time. However, while FITS is
compact, its ability to encode the coordinate frame and pixels, means
that data from different observational instruments can be stored in this
format and relationships between data from different instruments can be
related, rendering manual and error-prone procedures for conforming
images obsolete.

\subsection{High-energy physics (HEP)}\label{high-energy-physics-hep}

Because data collection is centralized, standards to collect and store
HEP data have been established and the adoption of these standards in
data analysis has high penetration (Basaglia et al. 2023). A top-down
approach is taken so that within every large collaboration standards are
enforced, and this adoption is centrally managed. Access to raw data is
essentially impossible, and making it publicly available is both
technically very hard and potentially ill-advised. Therefore, analysis
tools are tuned specifically to the standards. Incentives to use the
standards are provided by funders that require data management plans
that specify how the data is shared.

\subsection{Neuroscience}\label{neuroscience}

In contrast to astronomy and HEP, Neuroscience has traditionally been a
``cottage industry'', where individual labs have generated experimental
data designed to answer specific experimental questions. While this
model still exists, the field has also seen the emergence of new modes
of data production that focus on generating large shared datasets
designed to answer many different questions, more akin to the data
generated in large astronomy data collection efforts (Koch and Clay Reid
2012). This change has been brought on through a combination of
technical advances in data acquisition techniques, which now generate
large and very high-dimensional/information-rich datasets, cultural
changes, which have ushered in new norms of transparency and
reproducibility, and funding initiatives that have encouraged this kind
of data collection. However, because these changes are recent relative
to the other cases mentioned above, standards for data and metadata in
neuroscience have been prone to adopt many elements of modern OSS
development. Two salient examples in neuroscience are the Neurodata
Without Borders file format for neurophysiology data (Rübel et al. 2022)
and the Brain Imaging Data Structure (BIDS) standard for neuroimaging
data (Gorgolewski et al. 2016). BIDS in particular owes some of its
success to the adoption of OSS development mechanisms (Poldrack et al.
2024). For example, small changes to the standard are managed through
the GitHub pull request mechanism; larger changes are managed through a
a BIDS Enhancement Proposal (BEP) process that is directly inspired by
the Python programming language community's Python Enhancement Proposal
procedure, which used to introduce new ideas into the language. Though
the BEP mechanism takes a slightly different technical approach, it
tries to emulate the open-ended and community-driven aspects of Python
development to accept contributions from a wide range of stakeholders
and tap a broad base of expertise.

\subsection{Automated discovery}\label{automated-discovery}

\subsection{Citizen science}\label{citizen-science}

\section{Cross-sector interactions}\label{cross-sector-interactions}

The importance of standards stems not only from discussions within
research fields about how research can best be conducted to take
advantage of existing and growing datasets, but also arises from
interactions with other sectors. Several different kinds of cross-sector
interactions can be defined as having important impact on the
development of open-source standards.

\subsection{Governmental
policy-setting}\label{governmental-policy-setting}

The development of open practices in research has entailed an ongoing
interaction and dialogue with various governmental bodies that set
policies for research. For example, for research that is funded by the
public, this entails an ongoing series of policy discussions that
address the interactions between research communities and the general
public. One way in which this manifests in the United States
specifically is in memos issued by the directors of the White House
Office of Science and Technology Policy (OSTP), James Holdren (in 1) and
Alondra Nelson (in 2022). While these memos focused primarily on making
peer-reviewed publications funded by the US Federal government available
to the general public, they also lay an increasingly detailed path
toward the publication and general availability of the data that is
collected in research that is funded by the US government. The general
guidance and overall spirit of these memos dovetail with more specific
policy guidance related to data and metadata standards. For example, the
importance of standards was underscored in a recent report by the
Subcommittee on Open Science of the National Science and Technology
Council on the ``Desirable characteristics of data repositories for
federally funded research'' (The National Science and Technology Council
2022). The report explicitly called out the importance of
``allow{[}ing{]} datasets and metadata to be accessed, downloaded, or
exported from the repository in widely used, preferably non-proprietary,
formats consistent with standards used in the disciplines the repository
serves.'' This highlights the need for data and metadata standards
across a variety of different kinds of data. In addition, a report from
the National Institute of Standards and Technology on ``U.S. Leadership
in AI: A Plan for Federal Engagement in Developing Technical Standards
and Related Tools'' emphasized that -- specifically for the case of AI
-- ``U.S. government agencies should prioritize AI standards efforts
that are {[}\ldots{]} Consensus-based, {[}\ldots{]} Inclusive and
accessible, {[}\ldots{]} Multi-path, {[}\ldots{]} Open and transparent,
{[}\ldots{]} and {[}that{]} result in globally relevant and
non-discriminatory standards\ldots{}'' (National Institute of Standards
and Technology 2019). The converging characteristics of standards that
arise from these reports suggest that considerable thought needs to be
given to how standards arise so that these goals are achieved.

A compelling road map towards implementation and adoption of
community-developed standards is offered in a blog post authored by the
Center for Open Science's Brian Nosek, entitled ``Strategy for Culture
Change'' (Nosek, n.d.). The core idea is that affecting a turn toward
open science requires an alignment of not only incentives and values,
but also technical infrastructure and user experience. A sociotechnical
bridge between these pieces, which make adoption of standards possible,
and maybe even easy, and the policy goals, arises from a community of
practice that makes adoption of standards normative. Once all of these
pieces are in place, making adoption of open science standards required
becomes more straightforward and less onerous.

\subsection{Funding}\label{funding}

While government-set policy is primarily directed towards research that
is funded through governmental funding agencies, there are other ways in
which funding relates to the development of open-source standards. One
way is in funding the development of these standards. For example, the
National Institutes of Health have provided some of the funding for the
development of the Brain Imaging Data Structure standard in
neuroscience.

\section{Recommendations for open-source data and metadata
standards}\label{recommendations-for-open-source-data-and-metadata-standards}

In conclusion of this report, we propose the following recommendations:

\subsection{Funding or Grantmaking
entities:}\label{funding-or-grantmaking-entities}

\subsubsection{Fund Data Standards
Development}\label{fund-data-standards-development}

While some funding agencies already support standards development as
part of the development of informatics infrastructures, data standards
development should be seen as integral to science innovation and
earmarked for funding in research grants, not only in specialized
contexts. Funding models should encourage the development and adoption
of standards, and fund associated community efforts and tools for this.
The OSS model is seen as a particularly promising avenue for an
investment of resources, because it builds on previously-developed
procedures and technical infrastructure and because it provides avenues
for community input along the way. The clarity offered by procedures for
enhancement proposals and semantic versioning schemes adopted in
standards development offer avenues for a range of stakeholders to
propose to funding bodies well-defined contributions to large and
field-wide standards efforts.

\subsubsection{Invest in Data Stewards Recognize data stewards as a
distinct role
in}\label{invest-in-data-stewards-recognize-data-stewards-as-a-distinct-role-in}

research and science investment. Set up programs for training for data
stewards and invest in career paths that encourage this role. Initial
proposals for the curriculum and scope of the role have already been
proposed (e.g., in (Mons 2018))

\subsubsection{Review Data Standards
Pathways}\label{review-data-standards-pathways}

Invest in programs that examine retrospective pathways for establishing
data standards. Encourage publication of lifecycles for successful data
standards. Lifecycle should include process, creators, affiliations,
grants, and adoption journeys. Make this documentation step integral to
the work of standards creators and granting agencies. Retrocactively
document \#3 for standards such as CF(climate science), NASA genelab
(space omics), OpenGIS (geospatial), DICOM (medical imaging), GA4GH
(genomics), FITS (astronomy), Zarr (domain agnostic n-dimensional
arrays)\ldots{} ?

\subsubsection{Establish Governance}\label{establish-governance}

Establish governance for standards creation and adoption, especially for
communities beyond a certain size that need to converge toward a new
standard or rely on an existing standard. Review existing governance
practices such as
\href{https://www.theopensourceway.org/the_open_source_way-guidebook-2.0.html\#_project_and_community_governance}{TheOpenSourceWay}.
Data management plans should promote the sharing of not only data, but
also metadata and descriptions of how to use it.

\subsubsection{Program Manage Cross Sector
alliances}\label{program-manage-cross-sector-alliances}

Encourage cross sector and cross domain alliances that can impact
successful standards creation. Invest in robust program management of
these alliances to align pace and create incentives (for instance via
Open Source Program Office / OSPO efforts). Similar to program officers
at funding agencies, standards evolution need sustained PM efforts.
Multi company partnerships should include strategic initiatives for
standard establishment
e.g.~\href{https://www.pistoiaalliance.org/news/press-release-pistoia-alliance-launches-idmp-1-0/}{Pistoiaalliance}.

\subsubsection{Curriculum Development}\label{curriculum-development}

Stakeholder organizations should invest in training grants to establish
curriculum for data and metadata standards education.

\subsection{Science and Technology
Communities:}\label{science-and-technology-communities}

\subsubsection{User Driven Development}\label{user-driven-development}

Standards should be needs-driven and developed in close collaboration
with users. Changes and enhancements should be in response to community
feedback.

\subsubsection{Meta-Standards
development}\label{meta-standards-development}

Develop meta-standards or standards-of-standards. These are descriptions
of cross-cutting best-practices and can be used as a basis of the
analysis or assessment of an existing standard, or as guidelines to
develop new standards. For instance, barriers to adopting a data
standard irrespective of team size and technological capabilities should
be considered. Meta standards should include formalization for
versioning of standards \& interaction with related software. Naming of
standards should aid marketing and adoption.

\subsubsection{Ontology Development}\label{ontology-development}

Create ontology for standards process such as top down vs bottom up,
minimum number of datasets, community size. Examine schema.org (w3c),
PEP (Python), CDISC (FDA).

\subsubsection{Formalization Guidelines}\label{formalization-guidelines}

Amplify formalization/guidelines on how to create standards (example
metadata schema specifications using \href{https://linkml.io}{LinkML}.

\subsubsection{Landscape and Failure
Analysis}\label{landscape-and-failure-analysis}

Before establishing a new standard, survey and document failure of
current standards for a specific dataset / domain. Use resources such as
\href{https://fairsharing.org/}{Fairsharing} or
\href{https://www.dcc.ac.uk/guidance/standards}{Digital Curation
Center}.

\subsubsection{Machine Readability}\label{machine-readability}

Development of standards should be coupled with development of
associated software. Make data standards machine readable, and software
creation an integral part of establishing a standard's schema e.g.~For
identifiers for a person using CFF in citations, cffconvert software
makes the CFF standard usable and useful. Additionally, standards
evolution should maintain software compatibility, and ability to
translate and migrate between standards.

\section{Acknowledgements}\label{acknowledgements}

This report was produced following a
\href{https://uwescience.github.io/2024-open-source-standards-workshop/}{workshop
held at NSF headquarters in Alexandria, VA on April 8th-9th, 2024}. We
would like to thank the speakers and participants in this workshop for
the time and thought that they put into the workshop.

The workshop and this report were funded through
\href{https://www.nsf.gov/awardsearch/showAward?AWD_ID=2334483&HistoricalAwards=false}{NSF
grant \#2334483} from the NSF
\href{https://new.nsf.gov/funding/opportunities/pathways-enable-open-source-ecosystems-pose}{Pathways
to Enable Open-Source Ecosystems (POSE)} program.

\section*{References}\label{references}
\addcontentsline{toc}{section}{References}

\phantomsection\label{refs}
\begin{CSLReferences}{1}{0}
\bibitem[\citeproctext]{ref-Basaglia2023-dq}
Basaglia, T, M Bellis, J Blomer, J Boyd, C Bozzi, D Britzger, S Campana,
et al. 2023. {``Data Preservation in High Energy Physics.''} \emph{The
European Physical Journal C} 83 (9): 795.

\bibitem[\citeproctext]{ref-Gorgolewski2016BIDS}
Gorgolewski, Krzysztof J, Tibor Auer, Vince D Calhoun, R Cameron
Craddock, Samir Das, Eugene P Duff, Guillaume Flandin, et al. 2016.
{``The {Brain} {Imaging} {Data} {Structure}, a Format for Organizing and
Describing Outputs of Neuroimaging Experiments.''} \emph{Sci Data} 3
(June): 160044. \url{https://www.nature.com/articles/sdata201644}.

\bibitem[\citeproctext]{ref-Koch2012-ve}
Koch, Christof, and R Clay Reid. 2012. {``Observatories of the Mind.''}
\url{http://dx.doi.org/10.1038/483397a}.

\bibitem[\citeproctext]{ref-Mons2018DataStewardshipBook}
Mons, Barend. 2018. \emph{Data Stewardship for Open Science:
Implementing FAIR Principles}. 1st ed. Vol. 1. Milton: CRC Press.
\url{https://doi.org/10.1201/9781315380711}.

\bibitem[\citeproctext]{ref-NIST2019}
National Institute of Standards and Technology. 2019. {``{U.S}.
{LEADERSHIP} {IN} {AI}: A Plan for Federal Engagement in Developing
Technical Standards and Related Tools.''}

\bibitem[\citeproctext]{ref-Nosek2019CultureChange}
Nosek, Brian. n.d. {``Strategy for Culture Change.''}
\url{https://www.cos.io/blog/strategy-for-culture-change}.

\bibitem[\citeproctext]{ref-Poldrack2024BIDS}
Poldrack, Russell A, Christopher J Markiewicz, Stefan Appelhoff, Yoni K
Ashar, Tibor Auer, Sylvain Baillet, Shashank Bansal, et al. 2024. {``The
Past, Present, and Future of the Brain Imaging Data Structure
({BIDS}).''} \emph{ArXiv}, January.

\bibitem[\citeproctext]{ref-Rubel2022NWB}
Rübel, Oliver, Andrew Tritt, Ryan Ly, Benjamin K Dichter, Satrajit
Ghosh, Lawrence Niu, Pamela Baker, et al. 2022. {``The Neurodata Without
Borders Ecosystem for Neurophysiological Data Science.''} \emph{Elife}
11 (October).

\bibitem[\citeproctext]{ref-nstc2022desirable}
The National Science and Technology Council. 2022. {``Desirable
Characteristics of Data Repositories for Federally Funded Research.''}
\emph{Executive Office of the President of the United States, Tech.
Rep}.

\bibitem[\citeproctext]{ref-wells1979fits}
Wells, Donald Carson, and Eric W Greisen. 1979. {``FITS-a Flexible Image
Transport System.''} In \emph{Image Processing in Astronomy}, 445.

\bibitem[\citeproctext]{ref-Wilkinson2016FAIR}
Wilkinson, Mark D, Michel Dumontier, I Jsbrand Jan Aalbersberg,
Gabrielle Appleton, Myles Axton, Arie Baak, Niklas Blomberg, et al.
2016. {``The {FAIR} Guiding Principles for Scientific Data Management
and Stewardship.''} \emph{Sci Data} 3 (March): 160018.

\end{CSLReferences}



\end{document}
