% Options for packages loaded elsewhere
\PassOptionsToPackage{unicode}{hyperref}
\PassOptionsToPackage{hyphens}{url}
\PassOptionsToPackage{dvipsnames,svgnames,x11names}{xcolor}
%
\documentclass[
  letterpaper,
  DIV=11,
  numbers=noendperiod]{scrartcl}

\usepackage{amsmath,amssymb}
\usepackage{iftex}
\ifPDFTeX
  \usepackage[T1]{fontenc}
  \usepackage[utf8]{inputenc}
  \usepackage{textcomp} % provide euro and other symbols
\else % if luatex or xetex
  \usepackage{unicode-math}
  \defaultfontfeatures{Scale=MatchLowercase}
  \defaultfontfeatures[\rmfamily]{Ligatures=TeX,Scale=1}
\fi
\usepackage{lmodern}
\ifPDFTeX\else  
    % xetex/luatex font selection
\fi
% Use upquote if available, for straight quotes in verbatim environments
\IfFileExists{upquote.sty}{\usepackage{upquote}}{}
\IfFileExists{microtype.sty}{% use microtype if available
  \usepackage[]{microtype}
  \UseMicrotypeSet[protrusion]{basicmath} % disable protrusion for tt fonts
}{}
\makeatletter
\@ifundefined{KOMAClassName}{% if non-KOMA class
  \IfFileExists{parskip.sty}{%
    \usepackage{parskip}
  }{% else
    \setlength{\parindent}{0pt}
    \setlength{\parskip}{6pt plus 2pt minus 1pt}}
}{% if KOMA class
  \KOMAoptions{parskip=half}}
\makeatother
\usepackage{xcolor}
\setlength{\emergencystretch}{3em} % prevent overfull lines
\setcounter{secnumdepth}{5}
% Make \paragraph and \subparagraph free-standing
\ifx\paragraph\undefined\else
  \let\oldparagraph\paragraph
  \renewcommand{\paragraph}[1]{\oldparagraph{#1}\mbox{}}
\fi
\ifx\subparagraph\undefined\else
  \let\oldsubparagraph\subparagraph
  \renewcommand{\subparagraph}[1]{\oldsubparagraph{#1}\mbox{}}
\fi


\providecommand{\tightlist}{%
  \setlength{\itemsep}{0pt}\setlength{\parskip}{0pt}}\usepackage{longtable,booktabs,array}
\usepackage{calc} % for calculating minipage widths
% Correct order of tables after \paragraph or \subparagraph
\usepackage{etoolbox}
\makeatletter
\patchcmd\longtable{\par}{\if@noskipsec\mbox{}\fi\par}{}{}
\makeatother
% Allow footnotes in longtable head/foot
\IfFileExists{footnotehyper.sty}{\usepackage{footnotehyper}}{\usepackage{footnote}}
\makesavenoteenv{longtable}
\usepackage{graphicx}
\makeatletter
\def\maxwidth{\ifdim\Gin@nat@width>\linewidth\linewidth\else\Gin@nat@width\fi}
\def\maxheight{\ifdim\Gin@nat@height>\textheight\textheight\else\Gin@nat@height\fi}
\makeatother
% Scale images if necessary, so that they will not overflow the page
% margins by default, and it is still possible to overwrite the defaults
% using explicit options in \includegraphics[width, height, ...]{}
\setkeys{Gin}{width=\maxwidth,height=\maxheight,keepaspectratio}
% Set default figure placement to htbp
\makeatletter
\def\fps@figure{htbp}
\makeatother
% definitions for citeproc citations
\NewDocumentCommand\citeproctext{}{}
\NewDocumentCommand\citeproc{mm}{%
  \begingroup\def\citeproctext{#2}\cite{#1}\endgroup}
\makeatletter
 % allow citations to break across lines
 \let\@cite@ofmt\@firstofone
 % avoid brackets around text for \cite:
 \def\@biblabel#1{}
 \def\@cite#1#2{{#1\if@tempswa , #2\fi}}
\makeatother
\newlength{\cslhangindent}
\setlength{\cslhangindent}{1.5em}
\newlength{\csllabelwidth}
\setlength{\csllabelwidth}{3em}
\newenvironment{CSLReferences}[2] % #1 hanging-indent, #2 entry-spacing
 {\begin{list}{}{%
  \setlength{\itemindent}{0pt}
  \setlength{\leftmargin}{0pt}
  \setlength{\parsep}{0pt}
  % turn on hanging indent if param 1 is 1
  \ifodd #1
   \setlength{\leftmargin}{\cslhangindent}
   \setlength{\itemindent}{-1\cslhangindent}
  \fi
  % set entry spacing
  \setlength{\itemsep}{#2\baselineskip}}}
 {\end{list}}
\usepackage{calc}
\newcommand{\CSLBlock}[1]{\hfill\break\parbox[t]{\linewidth}{\strut\ignorespaces#1\strut}}
\newcommand{\CSLLeftMargin}[1]{\parbox[t]{\csllabelwidth}{\strut#1\strut}}
\newcommand{\CSLRightInline}[1]{\parbox[t]{\linewidth - \csllabelwidth}{\strut#1\strut}}
\newcommand{\CSLIndent}[1]{\hspace{\cslhangindent}#1}

\KOMAoption{captions}{tableheading}
\makeatletter
\@ifpackageloaded{caption}{}{\usepackage{caption}}
\AtBeginDocument{%
\ifdefined\contentsname
  \renewcommand*\contentsname{Table of contents}
\else
  \newcommand\contentsname{Table of contents}
\fi
\ifdefined\listfigurename
  \renewcommand*\listfigurename{List of Figures}
\else
  \newcommand\listfigurename{List of Figures}
\fi
\ifdefined\listtablename
  \renewcommand*\listtablename{List of Tables}
\else
  \newcommand\listtablename{List of Tables}
\fi
\ifdefined\figurename
  \renewcommand*\figurename{Figure}
\else
  \newcommand\figurename{Figure}
\fi
\ifdefined\tablename
  \renewcommand*\tablename{Table}
\else
  \newcommand\tablename{Table}
\fi
}
\@ifpackageloaded{float}{}{\usepackage{float}}
\floatstyle{ruled}
\@ifundefined{c@chapter}{\newfloat{codelisting}{h}{lop}}{\newfloat{codelisting}{h}{lop}[chapter]}
\floatname{codelisting}{Listing}
\newcommand*\listoflistings{\listof{codelisting}{List of Listings}}
\makeatother
\makeatletter
\makeatother
\makeatletter
\@ifpackageloaded{caption}{}{\usepackage{caption}}
\@ifpackageloaded{subcaption}{}{\usepackage{subcaption}}
\makeatother
\makeatletter
\@ifpackageloaded{fontspec}{}{\usepackage{fontspec}}
\makeatother
\makeatletter
\@ifpackageloaded{draftwatermark}{}{\usepackage{draftwatermark}}
\makeatother
\makeatletter
\@ifpackageloaded{xcolor}{}{\usepackage{xcolor}}
\makeatother
\makeatletter
\@ifpackageloaded{forloop}{}{\usepackage{forloop}}
\makeatother
    \definecolor{watermark}{HTML}{000000}
    
    \newcounter{watermarkrow}
    \newcounter{watermarkcol}

    \DraftwatermarkOptions{
      text={
        \begin{tabular}{c}
          \forloop{watermarkrow}{0}{\value{watermarkrow} < 50}{
            \forloop{watermarkcol}{0}{\value{watermarkcol} < 10}{
              { DRAFT}\hspace{4.000000em}
            }
            \\[4.000000em]
          }
        \end{tabular}
      },
      fontsize=1.000000em,
      angle=15.000000,
      color=watermark!10
    }
    
\ifLuaTeX
  \usepackage{selnolig}  % disable illegal ligatures
\fi
\usepackage{bookmark}

\IfFileExists{xurl.sty}{\usepackage{xurl}}{} % add URL line breaks if available
\urlstyle{same} % disable monospaced font for URLs
\hypersetup{
  pdftitle={Towards an open-source model for data and metadata standards},
  colorlinks=true,
  linkcolor={blue},
  filecolor={Maroon},
  citecolor={Blue},
  urlcolor={Blue},
  pdfcreator={LaTeX via pandoc}}

\title{Towards an open-source model for data and metadata standards}
\author{Ariel Rokem \and Vani Mandava}
\date{}

\begin{document}
\maketitle

\section{Abstract}\label{abstract}

Recent progress in machine learning and artificial intelligence promises
to advance research and understanding across a wide range of fields and
activities. In tandem, increased awareness of the importance of open
data for reproducibility and scientific transparency is making inroads
in fields that have not traditionally produced large publicly available
datasets. Data sharing requirements from publishers and funders, as well
as from other stakeholders, have also created pressure to make datasets
with research and/or public interest value available through digital
repositories. However, to make the best use of existing data, and
facilitate the creation of useful future datasets, robust, interoperable
and usable standards need to evolve and adapt over time. The open-source
development model provides significant potential benefits to the process
of standard creation and adaptation. In particular, the development and
adaptation of standards can use long-standing socio-technical processes
that have been key to managing the development of software, and allow
incorporating broad community input into the formulation of these
standards. By adhering to open-source standards to formal descriptions
(e.g., by implementing schemata for standard specification, and/or by
implementing automated standard validation), processes such as automated
testing and continuous integration, which have been important in the
development of open-source software, can be adopted in defining data and
metadata standards as well. Similarly, open-source governance provides a
range of stakeholders a voice in the development of standards,
potentially enabling use cases and concerns that would not be taken into
account in a top-down model of standards development. On the other hand,
open-source models carry unique risks that need to be incorporated into
the process.

\section{Introduction}\label{sec-intro}

Data-intensive discovery has become an important mode of knowledge
production across many research fields and it is having a significant
and broad impact across all of society. This is becoming increasingly
salient as recent developments in machine learning and artificial
intelligence (AI) promise to increase the value of large,
multi-dimensional, heterogeneous data sources. Coupled with these new
machine learning techniques, these datasets can help us understand
everything from the cellular operations of the human body, through
business transactions on the internet, to the structure and history of
the universe. However, the development of new machine learning methods
and data-intensive discovery more generally depends on Findability,
Accessibility, Interoperability and Reusability (FAIR) of data
(Wilkinson et al. 2016). One of the main mechanisms through which the
FAIR principles are promoted is the development of \emph{standards} for
data and metadata. Standards can vary in the level of detail and scope,
and encompass such things as \emph{file formats} for the storage of
certain data types, \emph{schemas} for databases that organize data,
\emph{ontologies} to describe and organize metadata in a manner that
connects it to field-specific meaning, as well as mechanisms to describe
\emph{provenance} of analysis products.

Community-driven development of robust, adaptable and useful standards
draws significant inspiration from the development of open-source
software (OSS) and has many parallels and overlaps with OSS development.
OSS has a long history going back to the development of the Unix
operating system in the late 1960s. Over the time since its inception,
the large community of developers and users of OSS have developed a host
of socio-technical mechanisms that support the development and use of
OSS. For example, the Open Source Initiative (OSI), a non-profit
organization that was founded in the 1990s developed a set of guidelines
for licensing of OSS that is designed to protect the rights of
developers and users. On the more technical side, tools such as the Git
Source-code management system support open-source development workflows
that can be adopted in the development of standards. Governance
approaches have been honed to address the challenges of managing a range
of stakeholder interests and to mediate between large numbers of
weakly-connected individuals that contribute to OSS. When these social
and technical innovations are put together they enable a host of
positive defining features of OSS, such as transparency, collaboration,
and decentralization. These features allow OSS to have a remarkable
level of dynamism and productivity, while also retaining the ability of
a variety of stakeholders to guide the evolution of the software to take
their needs and interests into account.

Data and metadata standards that adopt tools and practices of OSS
(``open-source standards'' henceforth) stand to reap many of the
benefits that the OSS model has provided in the development of other
technologies. The present report explore how OSS processes and tools
have affected the development of data and metadata standards. The report
will triangulate common features of a variety of use cases; it will
identify some of the challenges and pitfalls of this mode of standards
development, with a particular focus on cross-sector interactions; and
it will make recommendations for future developments and policies that
can help this mode of standards development thrive and reach its full
potential.

\section{Use cases}\label{sec-use-cases}

To understand how OSS development practices affect the development of
data and metadata standards, it is informative to demonstrate this
cross-fertilization through a few use cases. As we will see in these
examples, some fields, such as astronomy, high-energy physics and earth
sciences have a relatively long history of shared data resources from
organizations such as LSST, CERN, and NASA, while other fields have only
relatively recently become aware of the value of data sharing and its
impact. These disparate histories inform how standards have evolved and
how OSS practices have pervaded their development.

\subsection{Astronomy}\label{astronomy}

One prominent example of a community-driven standard is the FITS
(Flexible Image Transport System) file format standard, which was
developed in the late 1970s and early 1980s (Wells and Greisen 1979),
and has been adopted worldwide for astronomy data preservation and
exchange. Essentially every software platform used in astronomy reads
and writes the FITS format. It was developed by observatories in the
1980s to store image data in the visible and x-ray spectrum. It has been
endorsed by IAU, as well as funding agencies. Though the format has
evolved over time, ``once FITS, always FITS''. That is, the format
cannot be evolved to introduce changes that break backwards
compatibility. Among the features that make FITS so durable is that it
was designed originally to have a very restricted metadata schema. That
is, FITS records were designed to be the lowest common denominator of
word lengths in computer systems at the time. However, while FITS is
compact, its ability to encode the coordinate frame and pixels, means
that data from different observational instruments can be stored in this
format and relationships between data from different instruments can be
related, rendering manual and error-prone procedures for conforming
images obsolete.

\subsection{High-energy physics (HEP)}\label{high-energy-physics-hep}

Because data collection is centralized, standards to collect and store
HEP data have been established and the adoption of these standards in
data analysis has high penetration (Basaglia et al. 2023). A top-down
approach is taken so that within every large collaboration standards are
enforced, and this adoption is centrally managed. Access to raw data is
essentially impossible, and making it publicly available is both
technically very hard and potentially ill-advised. Therefore, analysis
tools are tuned specifically to the standards. Incentives to use the
standards are provided by funders that require data management plans
that specify how the data is shared.

\subsection{Earth sciences}\label{earth-sciences}

XXX

\subsection{Neuroscience}\label{neuroscience}

In contrast to astronomy and HEP, Neuroscience has traditionally been a
``cottage industry'', where individual labs have generated experimental
data designed to answer specific experimental questions. While this
model still exists, the field has also seen the emergence of new modes
of data production that focus on generating large shared datasets
designed to answer many different questions, more akin to the data
generated in large astronomy data collection efforts (Koch and Clay Reid
2012). This change has been brought on through a combination of
technical advances in data acquisition techniques, which now generate
large and very high-dimensional/information-rich datasets, cultural
changes, which have ushered in new norms of transparency and
reproducibility, and funding initiatives that have encouraged this kind
of data collection. However, because these changes are recent relative
to the other cases mentioned above, standards for data and metadata in
neuroscience have been prone to adopt many elements of modern OSS
development. Two salient examples in neuroscience are the Neurodata
Without Borders file format for neurophysiology data (Rübel et al. 2022)
and the Brain Imaging Data Structure (BIDS) standard for neuroimaging
data (Gorgolewski et al. 2016). BIDS in particular owes some of its
success to the adoption of OSS development mechanisms (Poldrack et al.
2024). For example, small changes to the standard are managed through
the GitHub pull request mechanism; larger changes are managed through a
BIDS Enhancement Proposal (BEP) process that is directly inspired by the
Python programming language community's Python Enhancement Proposal
procedure, which isused to introduce new ideas into the language. Though
the BEP mechanism takes a slightly different technical approach, it
tries to emulate the open-ended and community-driven aspects of Python
development to accept contributions from a wide range of stakeholders
and tap a broad base of expertise.

\subsection{Automated discovery}\label{automated-discovery}

\subsection{Community science}\label{community-science}

Another interesting use case for open-source standards is
community/citizen science. This approach, which has grown Here,
standards are needed to facilitate interactions between an in-group of
expert researchers who generate and curate data and a broader set of
out-group enthusiasts who would like to make meaningful contributions to
the science.

\section{Opportunities and risks for open-source
standards}\label{sec-challenges}

While open-source standards benefit from the technical and social
aspects of OSS, these tools and practices are associated with risks that
need to be mitigated.

\subsection{Flexibility vs.~Stability}\label{flexibility-vs.-stability}

One of the defining characteristics of OSS is its dynamism and its rapid
evolution. Because OSS can be used by anyone and, in most cases,
contributions can be made by anyone, innovations flow into OSS in a
bottom-up fashion from user/developers. Pathways to contribution by
members of the community are often well-defined: both from the technical
perspective (e.g., through a pull request on GitHub, or other similar
mechanisms), as well as from the social perspective (e.g., whether
contributors need to accept certain licensing conditions through a
contributor licensing agreement) and the socio-technical perspective
(e.g., how many people need to review a contribution, what are the
timelines for a contribution to be reviewed and accepted, what are the
release cycles of the software that make the contribution available to a
broader community of users, etc.). Similarly, open-source standards may
also find themselves addressing use cases and solutions that were not
originally envisioned through bottom-up contributions of members of a
research community to which the standard pertains. However, while this
dynamism provides an avenue for flexibility it also presents a source of
tension. This is because data and metadata standards apply to already
existing datasets, and changes may affect the compliance of these
existing datasets. Similarly, analysis technology stacks that are
developed based on an existing version of a standard have to adapt to
the introduction of new ideas and changes into a standard. Dynamic
changes of this sort therefore risk causing a loss of faith in the
standard by a user community, and migration away from the standard.
Similarly, if a standard evolves too rapidly, users may choose to stick
to an outdated version of a standard for a long time, creating strains
on the community of developers and maintainers of a standard who will
need to accommodate long deprecation cycles.

\subsection{Mismatches between standards developers and user
communities}\label{mismatches-between-standards-developers-and-user-communities}

In contrast to the OSS case, in open-source standards there is often an
inherent gap in both interest and ability to engage with the technical
details undergirding standards and their development between the core
developers of the standard and the users of the standard, which are the
broader field to which the standard pertains. This gap, in and of
itself, creates friction on the path to broad adoption and best
utilization of the standards. In extreme cases, the interests of
researchers and standards developers may even seem at odds, as
developers implement sophisticated mechanisms to automate the creation
and validation of the standard or advocate for more technically advanced
mechanisms for evolving the standard. These advanced capabilities offer
more robust development practices and consistency in cases where the
standards are complex and elaborate. They can ease the maintenance
burden of the standard. On the other hand, they may end up leaving
potential users sidelined in the development of the standard, and
limiting their ability to provide feedback about the practical
implications of changes to the standards.

\subsection{Unclear pathways for standards
success}\label{unclear-pathways-for-standards-success}

Standards typically develop organically through sustained and persistent
efforts from dedicated groups of data practitioners. These include
scientists and the broader ecosystem of data curators and users.
However, there is no playbook on the structure and components of a data
standard, or the pathway that moves the implementation of a specific
data architecture (e.g., a particular file format) to become a data
standard. As a result, data standardization lacks formal avenues for
success and recognition, for example through dedicated research grants
(and see Section~\ref{sec-cross-sector}). This hampers the long-term
trajectory that is needed in order to inculcate a standard into the
day-to-day practice of researchers.

\subsection{Cross-domain gaps}\label{cross-domain-gaps}

There is much to be gained from the development of standards that apply
in multiple different domains. For example, many research fields use
images as data and array-based computing that is applicable to images in
various research fields is at the core of many scientific computing
codes. This means that practitioners within any given field should be
motivated to draw on shared data standards and shared software
implementations of operations that are common across fields. On the
other hand, it is very hard to justify the investment in these common
resources. On the one hand, data standardization investment is even more
justified if the standard is generalizable beyond any specific science
domain. On the other hand, while the use cases are domain sciences
based, data standardization is seen as a data infrastructure and not a
science investment, reducing the immediate incentives for researchers to
engage with such efforts. This is exacerbated by science research
funding schemes that eschew generalized cross-domain solutions, and that
seek to generate tangible impact only with a specific domain.

\subsection{Data instrumentation
issues}\label{data-instrumentation-issues}

Where there is commercial interest in the development of data analysis
tools (e.g., in biomedical applications or applications were research
funding can be directed towards commercial solutions) there is an
incentive to create data formats and data analysis platforms that are
proprietary. This may drive innovative applications of scientific
measurements, but also creates sub-fields where scientific observations
are generated by proprietary instrumentation, due to these
commercialization or other profit-driven incentives. There is a lack of
regulatory oversight to adhere to available standards or evolve common
tools, limiting integration across different measurements. In cases
where a significant amount of data is already stored in proprietary
formats, significant data transformations may be required to get data to
a state that is amenable to open-source standards. In these sub-fields
there may also be a lack of incentive to set aside investment or
resources to invest in establishing open-source data standards, leaving
these sub-fields relatively siloed.

\subsubsection{Harnessing new computing paradigms and
technologies}\label{harnessing-new-computing-paradigms-and-technologies}

Open-source standards development faces the challenges of adapting to
new computing paradigms and technologies. Cloud computing provides a
particularly stark set of opportunities and challenges. On the one hand,
cloud computing offers practical solutions for many challenges of
contemporary data-driven research. For example, the scalability of cloud
resources addresses some of the challenges of the scale of data that is
produced by instruments in many fields. The cloud also makes data access
relatively straightforward, because of the ability to determine data
access permissions in a granular fashion. On the other hand, cloud
computing requires reinstrumenting many data formats. This is because
cloud data access patterns are fundamentally different from the ones
that are used in local posix-style file-systems. Suspicion of cloud
computing comes in two different flavors: the first by researchers and
administrators who may be wary of costs associated with cloud computing,
and especially with the difficulty of predicting these costs. Projects
such as NSF's Cloud Bank seek to mitigate some of these concerns, by
providing an additional layer of transparency into cloud costs (Norman
et al. 2021). The other type of objection relates to the fact that cloud
computing services, by their very nature, are closed ecosystems that
resist portability and interoperability. Some aspects of the services
are always going to remain hidden and privy only to the cloud computing
service provider. In this respect, cloud computing runs afoul of some of
the appealing aspects of OSS. That said, the development of ``cloud
native'' standards can provide significant benefits in terms of the
research that can be conducted. For example, NOAA plans to use cloud
computing for integration across the multiple disparate datasets that it
collects to build knowledge graphs that can be queried by researchers to
answer questions that can only be answered through this integration.
Putting all the data ``in one place'' should help with that. Adaptation
to the cloud in terms of data standards has driven development of new
file formats. A salient example is the ZARR format (Miles et al. 2024),
which supports random access into array-based datasets stored in cloud
object storage, facilitating scalable and parallelized computing on
these data. Indeed, data standards such as NWB (neuroscience) and OME
(microscopy) now use ZARR as a backend for cloud-based storage. In other
cases, file formats that were once not straightforward to use in the
cloud, such as HDF5 and TIFF have been adapted to cloud use (e.g.,
through the cloud-optimized geoTIFF format).

\subsection{Sustainability}\label{sustainability}

\subsection{The importance of automated
validation}\label{the-importance-of-automated-validation}

\section{Cross-sector interactions}\label{sec-cross-sector}

The importance of standards stems not only from discussions within
research fields about how research can best be conducted to take
advantage of existing and growing datasets, but also arises from
interactions with stakeholders in other sectors. Several different kinds
of cross-sector interactions can be defined as having an important
impact on the development of open-source standards.

\subsection{Governmental
policy-setting}\label{governmental-policy-setting}

The development of open practices in research has entailed an ongoing
interaction and dialogue with various governmental bodies that set
policies for research. For example, for research that is funded by the
public, this entails an ongoing series of policy discussions that
address the interactions between research communities and the general
public. One way in which this manifests in the United States
specifically is in memos issued by the directors of the White House
Office of Science and Technology Policy (OSTP), James Holdren (in 2013)
and Alondra Nelson (in 2022). While these memos focused primarily on
making peer-reviewed publications funded by the US Federal government
available to the general public, they also lay an increasingly detailed
path toward the publication and general availability of the data that is
collected in research that is funded by the US government. The general
guidance and overall spirit of these memos dovetail with more specific
policy guidance related to data and metadata standards. For example, the
importance of standards was underscored in a recent report by the
Subcommittee on Open Science of the National Science and Technology
Council on the ``Desirable characteristics of data repositories for
federally funded research'' (The National Science and Technology Council
2022). The report explicitly called out the importance of
``allow{[}ing{]} datasets and metadata to be accessed, downloaded, or
exported from the repository in widely used, preferably non-proprietary,
formats consistent with standards used in the disciplines the repository
serves.'' This highlights the need for data and metadata standards
across a variety of different kinds of data. In addition, a report from
the National Institute of Standards and Technology on ``U.S. Leadership
in AI: A Plan for Federal Engagement in Developing Technical Standards
and Related Tools'' emphasized that -- specifically for the case of AI
-- ``U.S. government agencies should prioritize AI standards efforts
that are {[}\ldots{]} Consensus-based, {[}\ldots{]} Inclusive and
accessible, {[}\ldots{]} Multi-path, {[}\ldots{]} Open and transparent,
{[}\ldots{]} and {[}that{]} result in globally relevant and
non-discriminatory standards\ldots{}'' (National Institute of Standards
and Technology 2019). The converging characteristics of standards that
arise from these reports suggest that considerable thought needs to be
given to how standards arise so that these goals are achieved.
Importantly, open-source standards seem to well-match at least some of
these characteristics.

The other side of policies is the implementation of these policies in
practice by developers of open-source standards and by the communities
to which the standards pertain. A compelling road map towards
implementation and adoption of open science practices in general and
open-source standards in particular is offered in a blog post authored
by the Center for Open Science's co-founder and executive director,
Brian Nosek, entitled ``Strategy for Culture Change'' (Nosek, n.d.). The
core idea is that affecting a turn toward open science requires an
alignment of not only incentives and values, but also technical
infrastructure and user experience. A sociotechnical bridge between
these pieces, which makes the adoption of standards possible, and maybe
even easy, and the policy goals, arises from a community of practice
that makes the adoption of standards \emph{normative}. Once all of these
pieces are in place, making adoption of open science standards
\emph{required} through policy becomes more straightforward and less
onerous.

\subsection{Funding}\label{funding}

Government-set policy intersects with funding considerations. This is
because it is primarily directed towards research that is funded through
governmental funding agencies. For example, high-level policy guidance
boils to practice in guidance for data management plans that are part of
funded research. In response to the policy guidance, these have become
increasingly more detailed and, for example, NSF- and NIH-funded
researchers are now required to both formulate their plans with more
clarity and increasingly also to share data using specified standards as
a condition for funding.

However, there are other ways in which funding relates to the
development of open-source standards. For example, through the BRAIN
Initiative, the National Institutes of Health have provided key funding
for the development of the Brain Imaging Data Structure standard in
neuroscience. Where large governmental funding agencies may not have the
resources or agility required to fund nascent or unconventional ways of
formulating standards, funding by non-governmental philanthropies and
other organizations can provide alternatives. One example (out of many)
is the Chan-Zuckerberg Initiative program for Essential Open Source
Software, which funds foundational open-source software projects that
have an application in biomedical sciences. Distinct from NIH funding,
however, some of this investment focuses on the development of OSS
practices. For example, funding to the Arrow project that focuses on
developing open-source software maintenance skills and practices, rather
than a specific biomedical application.

\subsection{Industry}\label{industry}

Interactions of data and meta-data standards with commercial interests
may provide specific sources of friction. This is because
proprietary/closed formats of data can create difficulty at various
transition points: from one instrument vendor to another, from data
producer to downstream recipient/user, etc. On the other hand, in some
cases, cross-sector collaborations with commercial entities may pave the
way to robust and useful standards. One example is the DICOM standard,
which is maintained by working groups that encompass commercial imaging
device vendors and researchers.

\section{Recommendations for open-source data and metadata
standards}\label{sec-recommendations}

In conclusion of this report, we propose the following recommendations:

\subsection{Policy-making and Funding
entities:}\label{policy-making-and-funding-entities}

\subsubsection{Fund Data Standards
Development}\label{fund-data-standards-development}

While some funding agencies already support standards development as
part of the development of informatics infrastructures, data standards
development should be seen as integral to science innovation and
earmarked for funding in research grants, not only in specialized
contexts. Funding models should encourage the development and adoption
of standards, and fund associated community efforts and tools for this.
The OSS model is seen as a particularly promising avenue for an
investment of resources, because it builds on previously-developed
procedures and technical infrastructure and because it provides avenues
for democratization of development processes and for community input
along the way. The clarity offered by procedures for enhancement
proposals and semantic versioning schemes adopted in standards
development offer avenues for a range of stakeholders to propose to
funding bodies well-defined contributions to large and field-wide
standards efforts (e.g., (Pestilli et al. 2021)).

\subsubsection{Invest in Data Stewards}\label{invest-in-data-stewards}

Advancing the development and adoption of open-source standards requires
the dissemination of knowledge to researchers in a variety of fields,
but this dissemination itself may not be enough without the fostering of
specialized expertise. Therefore, it is important to recognize
\emph{data stewards} as a distinct role in research. To truly support
experts whose role will be to develop, maintain, and facilitate the
adoption and use of open-source standards, it will be necessary to set
up programs for training for data stewards and invest in career paths
that encourage this role. Initial proposals for the curriculum and scope
of the role have already been proposed (e.g., in (Mons 2018)). In
addition, in order for these individuals to be able to make the best use
of open-source standards, it will be important for these individuals to
be facile in the methodology of OSS. This does not mean that they need
to become software engineers -- though there may be some overlap with
the role of research software engineers (Connolly et al. 2023) -- but
rather that they need to become familiar with those parts of the OSS
development life-cycle that are useful for development of open-source
standards.

\subsubsection{Review Data Standards
Pathways}\label{review-data-standards-pathways}

Invest in programs that examine retrospective pathways for establishing
data standards. Encourage publication of lifecycles for successful data
standards. Lifecycle should include process, creators, affiliations,
grants, and adoption journeys. Make this documentation step integral to
the work of standards creators and granting agencies. Retrocactively
document \#3 for standards such as CF(climate science), NASA genelab
(space omics), OpenGIS (geospatial), DICOM (medical imaging), GA4GH
(genomics), FITS (astronomy), Zarr (domain agnostic n-dimensional
arrays)\ldots{} ?

\subsubsection{Establish Governance}\label{establish-governance}

Establish governance for standards creation and adoption, especially for
communities beyond a certain size that need to converge toward a new
standard or rely on an existing standard. Review existing governance
practices such as
\href{https://www.theopensourceway.org/the_open_source_way-guidebook-2.0.html\#_project_and_community_governance}{TheOpenSourceWay}.
Data management plans should promote the sharing of not only data, but
also metadata and descriptions of how to use it.

\subsubsection{Program Manage Cross Sector
alliances}\label{program-manage-cross-sector-alliances}

Encourage cross-sector and cross-domain alliances that can impact
successful standards creation. Invest in robust program management of
these alliances to align pace and create incentives (for instance via
Open Source Program Office / OSPO efforts). Similar to program officers
at funding agencies, standards evolution need sustained PM efforts.
Multi company partnerships should include strategic initiatives for
standard establishment e.g.
\href{https://www.pistoiaalliance.org/news/press-release-pistoia-alliance-launches-idmp-1-0/}{Pistoiaalliance}.

\subsubsection{Curriculum Development}\label{curriculum-development}

Stakeholder organizations should invest in training grants to establish
curriculum for data and metadata standards education.

\subsection{Science and Technology
Communities:}\label{science-and-technology-communities}

\subsubsection{User-Driven Development}\label{user-driven-development}

Standards should be needs-driven and developed in close collaboration
with users. Changes and enhancements should be in response to community
feedback.

\subsubsection{Meta-Standards
development}\label{meta-standards-development}

Develop meta-standards or standards-of-standards. These are descriptions
of cross-cutting best-practices and can be used as a basis of the
analysis or assessment of an existing standard, or as guidelines to
develop new standards. For instance, barriers to adopting a data
standard irrespective of team size and technological capabilities should
be considered. Meta standards should include formalization for
versioning of standards \& interaction with related software. Naming of
standards should aid marketing and adoption.

\subsubsection{Ontology Development}\label{ontology-development}

Create ontology for standards process such as top down vs bottom up,
minimum number of datasets, community size. Examine schema.org (w3c),
PEP (Python), CDISC (FDA).

\subsubsection{Formalization Guidelines}\label{formalization-guidelines}

Amplify formalization/guidelines on how to create standards (example
metadata schema specifications using \href{https://linkml.io}{LinkML}.

\subsubsection{Landscape and Failure
Analysis}\label{landscape-and-failure-analysis}

Before establishing a new standard, survey and document failure of
current standards for a specific dataset / domain. Use resources such as
\href{https://fairsharing.org/}{Fairsharing} or
\href{https://www.dcc.ac.uk/guidance/standards}{Digital Curation
Center}.

\subsubsection{Machine Readability}\label{machine-readability}

Development of standards should be coupled with development of
associated software. Make data standards machine readable, and software
creation an integral part of establishing a standard's schema e.g.~For
identifiers for a person using CFF in citations, cffconvert software
makes the CFF standard usable and useful. Additionally, standards
evolution should maintain software compatibility, and ability to
translate and migrate between standards.

\section{Acknowledgements}\label{acknowledgements}

This report was produced following a
\href{https://uwescience.github.io/2024-open-source-standards-workshop/}{workshop
held at NSF headquarters in Alexandria, VA on April 8th-9th, 2024}. We
would like to thank the speakers and participants in this workshop for
the time and thought that they put into the workshop.

The workshop and this report were funded through
\href{https://www.nsf.gov/awardsearch/showAward?AWD_ID=2334483&HistoricalAwards=false}{NSF
grant \#2334483} from the NSF
\href{https://new.nsf.gov/funding/opportunities/pathways-enable-open-source-ecosystems-pose}{Pathways
to Enable Open-Source Ecosystems (POSE)} program.

\section*{References}\label{references}
\addcontentsline{toc}{section}{References}

\phantomsection\label{refs}
\begin{CSLReferences}{1}{0}
\bibitem[\citeproctext]{ref-Basaglia2023-dq}
Basaglia, T, M Bellis, J Blomer, J Boyd, C Bozzi, D Britzger, S Campana,
et al. 2023. {``Data Preservation in High Energy Physics.''} \emph{The
European Physical Journal C} 83 (9): 795.

\bibitem[\citeproctext]{ref-Connolly2023Software}
Connolly, Andrew, Joseph Hellerstein, Naomi Alterman, David Beck, Rob
Fatland, Ed Lazowska, Vani Mandava, and Sarah Stone. 2023. {``{Software}
{Engineering} {Practices} in {Academia}: Promoting the
3Rs---{Readability}, {Resilience}, and {Reuse}.''} \emph{Harvard Data
Science Review} 5 (2).

\bibitem[\citeproctext]{ref-Gorgolewski2016BIDS}
Gorgolewski, Krzysztof J, Tibor Auer, Vince D Calhoun, R Cameron
Craddock, Samir Das, Eugene P Duff, Guillaume Flandin, et al. 2016.
{``The {Brain} {Imaging} {Data} {Structure}, a Format for Organizing and
Describing Outputs of Neuroimaging Experiments.''} \emph{Sci Data} 3
(June): 160044. \url{https://www.nature.com/articles/sdata201644}.

\bibitem[\citeproctext]{ref-Koch2012-ve}
Koch, Christof, and R Clay Reid. 2012. {``Observatories of the Mind.''}
\url{http://dx.doi.org/10.1038/483397a}.

\bibitem[\citeproctext]{ref-zarr}
Miles, Alistair, jakirkham, M Bussonnier, Josh Moore, Dimitri
Papadopoulos Orfanos, Davis Bennett, David Stansby, et al. 2024.
{``Zarr-Developers/Zarr-Python: V3.0.0-Alpha.''} Zenodo.
\url{https://doi.org/10.5281/zenodo.11592827}.

\bibitem[\citeproctext]{ref-Mons2018DataStewardshipBook}
Mons, Barend. 2018. \emph{Data Stewardship for Open Science:
Implementing FAIR Principles}. 1st ed. Vol. 1. Milton: CRC Press.
\url{https://doi.org/10.1201/9781315380711}.

\bibitem[\citeproctext]{ref-NIST2019}
National Institute of Standards and Technology. 2019. {``{U.S}.
{LEADERSHIP} {IN} {AI}: A Plan for Federal Engagement in Developing
Technical Standards and Related Tools.''}

\bibitem[\citeproctext]{ref-Norman2021CloudBank}
Norman, Michael, Vince Kellen, Shava Smallen, Brian DeMeulle, Shawn
Strande, Ed Lazowska, Naomi Alterman, et al. 2021. {``{CloudBank:
Managed Services to Simplify Cloud Access for Computer Science Research
and Education}.''} In \emph{Practice and Experience in Advanced Research
Computing}. PEARC '21. New York, NY, USA: Association for Computing
Machinery. \url{https://doi.org/10.1145/3437359.3465586}.

\bibitem[\citeproctext]{ref-Nosek2019CultureChange}
Nosek, Brian. n.d. {``Strategy for Culture Change.''}
\url{https://www.cos.io/blog/strategy-for-culture-change}.

\bibitem[\citeproctext]{ref-pestilli2021community}
Pestilli, Franco, Russ Poldrack, Ariel Rokem, Theodore Satterthwaite,
Franklin Feingold, Eugene Duff, Cyril Pernet, Robert Smith, Oscar
Esteban, and Matt Cieslak. 2021. {``A Community-Driven Development of
the Brain Imaging Data Standard (BIDS) to Describe Macroscopic Brain
Connections.''} \emph{OSF}.

\bibitem[\citeproctext]{ref-Poldrack2024BIDS}
Poldrack, Russell A, Christopher J Markiewicz, Stefan Appelhoff, Yoni K
Ashar, Tibor Auer, Sylvain Baillet, Shashank Bansal, et al. 2024. {``The
Past, Present, and Future of the Brain Imaging Data Structure
({BIDS}).''} \emph{ArXiv}, January.

\bibitem[\citeproctext]{ref-Rubel2022NWB}
Rübel, Oliver, Andrew Tritt, Ryan Ly, Benjamin K Dichter, Satrajit
Ghosh, Lawrence Niu, Pamela Baker, et al. 2022. {``The Neurodata Without
Borders Ecosystem for Neurophysiological Data Science.''} \emph{Elife}
11 (October).

\bibitem[\citeproctext]{ref-nstc2022desirable}
The National Science and Technology Council. 2022. {``Desirable
Characteristics of Data Repositories for Federally Funded Research.''}
\emph{Executive Office of the President of the United States, Tech.
Rep}.

\bibitem[\citeproctext]{ref-wells1979fits}
Wells, Donald Carson, and Eric W Greisen. 1979. {``FITS-a Flexible Image
Transport System.''} In \emph{Image Processing in Astronomy}, 445.

\bibitem[\citeproctext]{ref-Wilkinson2016FAIR}
Wilkinson, Mark D, Michel Dumontier, I Jsbrand Jan Aalbersberg,
Gabrielle Appleton, Myles Axton, Arie Baak, Niklas Blomberg, et al.
2016. {``The {FAIR} Guiding Principles for Scientific Data Management
and Stewardship.''} \emph{Sci Data} 3 (March): 160018.

\end{CSLReferences}



\end{document}
